
\documentclass{uonmathreport}

% this allows one to include .jpg etc figures using pdflatex
% change the optional argument if you use dvips or others
\usepackage[pdftex]{graphicx}


% other packages that maybe of use include:
% hyperref, amsthm, xy, todonotes, showkeys, ...

% change to \PJS or \DIS or \HGDIS (for BSc and MPhil)
% or \MSc (for all Msc dissertations)
\MSc

% adjust the following
\title{An Adaptive Finite Element Method Approach to the Heat Equation and the Black-Scholes Equation}
\author{Thabo Miles 'Matli}
\academicyear{2017/2018}
\supervisor{Dr. Kris Van Der Zee}

% the following are irrelevant for Msc:
\assessmenttype{Review} % or Investigation
\projectcode{XX P99}

% the following are irrelevant for PJS, PJA, DIS and HG4DIS:
% Msc: change it to G1PMD and Pure Mathematics, etc ...
\msccode{G14SCD}
\msctitle{Scientific Computation}

% gives double-spacing
\linespread{1.6}
% the margins are set automatically. Do not make them smaller.

% put your own definitions and shorthands here
\newcommand{\ZZ}{\mathbb{Z}}

\begin{document}

\maketitle

\begin{abstract}
The abstract of the report goes here. The abstract should state the
topic(s) under investigation and the main results or
conclusions. Methods or approaches should be stated if this is
appropriate for the topic. The abstract should be self-contained,
concise and clear. The typical length is one paragraph.
\end{abstract}

% Table of contents
\setcounter{tocdepth}{2}  % this will list subsections, but not subsubsections
\tableofcontents 
\newpage

\section{Introduction} \label{sec:intro}

The introductory section goes here. And remember the introduction
is the last thing you write. 

The origin of the Finite Element Method (FEM) is generally agreed to be a paper by Courant \cite{courant1943} in 1943. Though initially obscure, it gained widespread usage in engineering as computing power became more cheaply available. Since then it has become increasingly more common in the natural sciences and more recently in financial industry \cite{topper2005option}. Though the Finite Difference Method is still overwhelming used to price options the FEM is now sometimes being instead. 

Though more technical than the Finite Difference Method, under certain circumstance the FEM has stark advantages. Two notable advantages are that the FEM is simpler to use for Partial Differential Equations (PDEs) with irregular shaped domains and that there is a very well understood theory of aposteriori errors. This theory of errors, which only requires knowledge of the estimated solution, allows for the FEM to be adapted during implementation. This adaptive methodology leads to a solution where errors are guaranteed to be within certain tolerances which allows the user to analyse the effects of changing parameters. Also, adaptivity leads to a solution that should be in some sense efficient as ideally maximum accuracy achieved for the minimum degrees of freedom. This concept of efficiency is what we look to investigate further in this paper.

The pioneering work of Babuska et al \cite{babuska1981posteriori} in the 1980s showed the first examples of how an aposteriori error estimate could be used to implement adaptive FEM. The research moved quickly and attempts at adaptive mesh refinement for parabolic PDEs began towards the end of the same decade see \cite{eriksson1991adaptive}, \cite{johnson1988error} and others. Despite the research into these methods and the adoption of adaptive FEM in science and engineering there are still some theoretical results outstanding. Convergence and optimal complexity have only been shown for linear elliptical PDEs and only quite recently \cite{morin2008basic}, \cite{siebert2011convergence}. Even these recent results only show that there is convergence to a solution and do not imply an order of convergence for the adaptive methods.

The rest of this dissertation comprises:
 
...

As show by Bergam \cite{Bergam} in their article.


% this is a comment in the file that won't appear in the output

The end of the introductory section would typically outline the
structure of the report. In this template, section \ref{sec:background}
gives the background of the topic, sections \ref{sec:my1} and
\ref{sec:my2} contain the bulk of the work and section
\ref{sec:conclusions} summarises and discusses what has been
achieved. Appendix \ref{app:rawdata} displays the raw data, and
certain technical calculations for section \ref{sec:my1} are deferred
to appendix \ref{app:calculations}.

\newpage

\section{The Linear Heat Equation} \label{sec:Heat Equation}

References can be for example
textbooks \cite{bott-tu,haw-ell,wolf,alling-greenleaf,hatcher},
conventional journal articles \cite{wheeler-geon,dewitt-can},
conventional journal articles that are also available at an e-print
server \cite{krasnov-louko,barrett-dawe}, electronic journal
articles \cite{poisson-livrev}, articles in conference proceedings
\cite{poisson-gr17}, PhD theses \cite{giulini-thesis,langlois-thesis}
or websites \cite{ligo-site}. This template orders the references by
their first citation, cites them by their number and keeps any
footnotes\footnote{Such as this.} separate from the references. Other
citation practices exist: Your supervisor can advise as to what is
appropriate for your topic.

\newpage

\section{The Finite Element Method for the Heat Equation} \label{sec:FEM}

This section is concerned with the discretisation of the Heat Equation by Finite Elements. Though not uncommon the FEM is not necessarily widely known by the average postgraduate mathematician. As such we will provide a fairly detailed description of the method and full derivation of the scheme. A very nice introduction to the subject is \cite{larson2013finite}.

The underlying practical steps to the method are:

\begin{enumerate}
\item Finding a variational form of the equation under consideration. This will allow us to look for a solution which satisfies the equation in some weak sense. This step requires the introduction of the function spaces where we will set our equation and look for its solution. We will see that these spaces are Sobolev spaces.

\item Dividing our domain $\Omega$, in our case just an interval, into elements.

\item Taking the variational form which we have at first defined in an infinite dimensional function space and approximate it on a finite dimensional sub-space. In our case the subspace will be piecewise polynomial functions defined on our subdivision.

\item Projecting the boundary conditions onto the finite dimensional space.
\end{enumerate}

\subsection{A subsection} \label{subsec:theory}

Subsections may be used. Use a clear structure in your report.

We denote the set of real numbers by
$\mathbb{R}$, the set of integers by $\ZZ$ and the set of complex
numbers by $\mathbb{C}$. Our analysis is based on the equation
$e^{\pi i} = -1$ and the relation
\begin{equation}
  \frac{2}{4} = \frac{1}{2}   \label{eq:myeq1}
\end{equation} % no empty line after this
which we verify in the appendix \ref{app:calculations}.
Useful consequences are
\begin{align}
  \frac{4}{8} &= \frac{1}{2} \\
  \frac{4}{12} + \frac{1}{\Gamma(s)}\int_0^{\infty} \frac{t^{s-1}}{e^t-1} dt
     &= \frac{1}{3} +\sum_{n=1}^{\infty} \frac{1}{n^s}\\
  \frac{2}{10} &= \frac{1}{5} 
\end{align}
For any $0\neq a\in \ZZ$, the equality
\begin{equation*} % * for no numbering
 \frac{2 a}{4 a} = \frac{1}{2}
\end{equation*}
follows from equation \eqref{eq:myeq1}.

\subsection{Another subsection} \label{subsec:application}

\subsubsection{A subsubsection} \label{subsubsec:red}

Sometimes subsubsections may be appropriate.

\subsubsection{Another subsubsection} \label{subsubsec:green}

This could contain a table of interesting numbers
\begin{center}
  \begin{tabular}{r|cccccc}
    $n$   & 1 & 2 & 3 & 4 & 5 & 6 \\ \hline
    $F_n$ & 1 & 1 & 2 & 3 & 5 & 8 \\
    $B_n$ & $\tfrac{1}{2}$ & $\tfrac{1}{6}$ & 0 & $-\tfrac{1}{30}$ & 0 &  $\tfrac{1}{42}$ \\
    $p_n$ & 2 & 3& 5& 7 & 11 & 13 \\
  \end{tabular}
\end{center}

\newpage

\section{Implementation} \label{sec:my2}

Graphics can be included. Figure \ref{fig:bsd} shows an example.
Learn about floats and pictures in the \LaTeX\ wikibook to place
the figures at the right place in the end.
%
\begin{figure}
 \begin{center}
 \end{center}
 \caption{Oh look, something happens here !}
 \label{fig:bsd}
\end{figure}

\newpage

\section{Conclusions} \label{sec:conclusions}

Further help on \LaTeX\ can be found easily on the internet. The \LaTeX\
wikibook\footnote{\tt http://en.wikibooks.org/wiki/LaTeX} contains a lot.
For instance you would find there how to type theorems and proofs nicely.
Or how to include source code written in some programming language like
matlab. There are long lists available with all sorts of common
mathematical symbols like $\xi$, $\nabla$, $\infty$, $\log$, $\iff$, etc.

\newpage

\appendix

\section{Raw data} \label{app:rawdata}

Material that needs to be included but would distract from the main
line of presentation can be put in appendices.
Examples of such material are raw
data, computing codes and details of calculations.


\section{Calculations for section \ref{sec:my1}} \label{app:calculations}

In this appendix we verify equation \eqref{eq:myeq1}.

\newpage

	\bibliography{References}
	\bibliographystyle{plain}

\end{document}
