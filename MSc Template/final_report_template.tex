
\documentclass{uonmathreport}

% this allows one to include .jpg etc figures using pdflatex
% change the optional argument if you use dvips or others
\usepackage[pdftex]{graphicx}

\usepackage{amsthm}
\usepackage{bigints}
\usepackage{relsize}
\usepackage{amsmath}
\usepackage{tikz}
\usepackage{listings}
\lstset {backgroundcolor=\color{black!5}, basicstyle=\footnotesize, stringstyle=\color{red}, commentstyle=\color{green!50!black}, basicstyle=\footnotesize\ttfamily, keywordstyle=\color{blue}, 
}
\usepackage{xcolor}
 
\theoremstyle{definition}
\newtheorem{definition}{Definition}[section]

\theoremstyle{problem}
\newtheorem{problem}{Problem}[section]

\theoremstyle{theorem}
\newtheorem{theorem}{Theorem}[section]

\def\realnumbers{\mathbb{R}}

\DeclareMathSizes{12}{12}{10}{10}


% other packages that maybe of use include:
% hyperref, amsthm, xy, todonotes, showkeys, ...

% change to \PJS or \DIS or \HGDIS (for BSc and MPhil)
% or \MSc (for all Msc dissertations)
\MSc

% adjust the following
\title{An Adaptive Finite Element Method Approach to the Heat Equation and the Black-Scholes Equation}
\author{Thabo Miles 'Matli}
\academicyear{2017/2018}
\supervisor{Dr. Kris Van Der Zee}

% the following are irrelevant for Msc:
\assessmenttype{Review} % or Investigation
\projectcode{XX P99}

% the following are irrelevant for PJS, PJA, DIS and HG4DIS:
% Msc: change it to G1PMD and Pure Mathematics, etc ...
\msccode{G14SCD}
\msctitle{Scientific Computation}

% gives double-spacing
\linespread{1.6}
% the margins are set automatically. Do not make them smaller.

% put your own definitions and shorthands here
\newcommand{\ZZ}{\mathbb{Z}}

\begin{document}

\maketitle

\begin{abstract}
The abstract of the report goes here. The abstract should state the
topic(s) under investigation and the main results or
conclusions. Methods or approaches should be stated if this is
appropriate for the topic. The abstract should be self-contained,
concise and clear. The typical length is one paragraph.
\end{abstract}

% Table of contents
\setcounter{tocdepth}{2}  % this will list subsections, but not subsubsections
\tableofcontents 
\newpage

\section{Introduction} \label{sec:intro}

The origin of the Finite Element Method (FEM) is generally agreed to be a paper by Courant \cite{courant1943} in 1943. Though initially obscure, it gained widespread usage in engineering as computing power became more cheaply available. Since then it has become increasingly more common in the natural sciences and more recently in financial industry \cite{topper2005option}. Though the Finite Difference Method is still overwhelming used to price options the FEM is now sometimes being instead. 

Though more technical than the Finite Difference Method, under certain circumstance the FEM has stark advantages. Two notable advantages are that the FEM is simpler to use for Partial Differential Equations (PDEs) with irregular shaped domains and that there is a very well understood theory of aposteriori errors. This theory of errors, which only requires knowledge of the estimated solution, allows for the FEM to be adapted during implementation. This adaptive methodology leads to a solution where errors are guaranteed to be within certain tolerances which allows the user to analyse the effects of changing parameters. Also, adaptivity leads to a solution that should be in some sense efficient as ideally maximum accuracy achieved for the minimum degrees of freedom. This concept of efficiency is what we look to investigate further in this paper.

The pioneering work of Babuska et al \cite{babuska1981posteriori} in the 1980s showed the first examples of how an aposteriori error estimate could be used to implement adaptive FEM. The research moved quickly and attempts at adaptive mesh refinement for parabolic PDEs began towards the end of the same decade see \cite{eriksson1991adaptive}, \cite{johnson1988error} and others. Despite the research into these methods and the adoption of adaptive FEM in science and engineering there are still some theoretical results outstanding. Convergence and optimal complexity have only been shown for linear elliptical PDEs and only quite recently \cite{morin2008basic}, \cite{siebert2011convergence}. Even these recent results only show that there is convergence to a solution and do not imply an order of convergence for the adaptive methods.

The rest of this dissertation comprises:
 
...

As show by Bergam \cite{Bergam} in their article.


% this is a comment in the file that won't appear in the output

The end of the introductory section would typically outline the
structure of the report. In this template, section \ref{sec:background}
gives the background of the topic, sections \ref{sec:my1} and
\ref{sec:my2} contain the bulk of the work and section
\ref{sec:conclusions} summarises and discusses what has been
achieved. Appendix \ref{app:rawdata} displays the raw data, and
certain technical calculations for section \ref{sec:my1} are deferred
to appendix \ref{app:calculations}.

\newpage

\section{The Linear Heat Equation} \label{sec:Heat Equation}

The Heat Equation provides a simple situation within which we can understand and demonstrate the implementation of the Finite Element Method and adaptive algorithms. Conversely it is fundamental enough that it would allow someone reading this dissertation to easily build on our findings to access a different but connected problem for example the Black-Scholes equation. It can be thought of as a prototypical parabolic equation and this was our motivation for studying it.

In this section we will describe the derivation of the Heat Equation in one dimension. We will first do this in the steady state i.e. where $\frac{du}{dt}=0$ and then extend this to the time dependent equation. We do this just to refresh the reader's knowledge of the properties of the equation. 

\subsection{The Heat Equation in Steady State} \label{subsec:Steady State}

The steady state equation in one dimension can be thought of as describing a thin rod of uniform material on the interval $I =$ [0, L]. As we are only considering the one dimensional problem heat is only conducted in the $x$ direction. The rod is heated by a source $f$ which we assume to have been acting continuously for long enough to reach the steady state. 

Let $q$ be the heat flux i.e flow of energy per unit of area per unit of time and let $S$ be the cross section of the rod. As flux is a vector quantity it needs a direction and we take this as the direction of x increasing. The first law of thermodynamics on conservation of energy tells us that the flow out of the rod must equal the flow in from the heat source. Hence we have:

\begin{equation}
q(L)S(L) - q(0)S(0) =  \int_I  f  dx	\label{eq:Conservation of Energy1}
\end{equation}

We now divide both sides of \eqref{eq:Conservation of Energy1} by L and take L $\rightarrow$ 0. Hence we have the differential equation:

\begin{equation}
(Sq)' = f	\label{eq:Conservation of Energy2}
\end{equation}

Employing Fouriers law which in this context can be understood as the flux being negatively proportional to the temperature gradient:

\begin{equation}
q = kT'	\label{eq:Fouriers Law}
\end{equation}

here k represents the heat conductivity of the rod.

Together \eqref{eq:Conservation of Energy2} and \eqref{eq:Fouriers Law} give us the Heat Equation:

\begin{equation}
-(SkT')'= f 	\label{eq:Steady State Heat Equation}
\end{equation}


\subsection{The Time Dependent Problem} \label{subsec:Time Dependent Heat Equation}

The time dependent problem is very similar to the steady state form. We introduce a function e which is the energy per unit length within the rod. We use the the conservation of energy principle again but this time we use the fact that the sum of the resultant heat flux is equal to the rate of change of internal energy. Here rate of change of $e$ is denoted by $e\dot{•}$

\begin{equation}
\int_I  \dot{e} dx =   q(0)S(0) - q(L)S(L)  \int_I  f  dx	\label{eq:Conservation of Energyt1}
\end{equation}

Once again dividing by L and letting L $\rightarrow$ 0.

\begin{equation}
\dot{e} + (Sq)' = f	\label{eq:Conservation of Energyt2}
\end{equation}

Finally assuming that the energy depends linearly on the the Temperature

\begin{equation}
e = mT	\label{eq:Energy and Heat}
\end{equation}

And using \eqref{eq:Fouriers Law} and combining \eqref{eq:Conservation of Energyt1} and \eqref{eq:Conservation of Energyt2} we have the transient Heat Equation which we will refer through out the rest of this paper simply as the Heat Equation:

 \begin{equation}
m\dot{T} - (SkT')' = f	\label{eq:Heat Equation}
\end{equation}


\clearpage
\subsection{Boundary Conditions} \label{subsec:Boundary Conditions}



\begin{figure}[h]
   \includegraphics[width=0.5\textwidth]{Heat1.pdf}
   \includegraphics[width=0.5\textwidth]{Heat2.pdf}
   
 \caption{$f= 0$, $m=1$, $S = 1$, $k = \pi^-2$}
 \label{fig:Heat1}
\end{figure}



\clearpage

\section{The Finite Element Method} \label{sec:FEM}

This section is concerned with the discretisation of the Heat Equation by Finite Elements. Though not uncommon, the FEM is not necessarily widely known by the average postgraduate mathematician. As such we will provide a fairly detailed description of the method and full derivation of the scheme. A very nice introduction to the subject if more detail is needed is \cite{larson2013finite}.

The underlying practical steps to the method are:

\begin{enumerate}
\item Finding a variational form of the equation we are looking to solve. This will allow us to look for a solution which satisfies the equation in some weak sense. This step requires the introduction of the function spaces in which we can study our  equation and look for its solution. We will see that these spaces are Sobolev spaces.

\item Creating the subdivision of our domain $\Omega$.

\item Taking the variational form, which we have at first defined in an infinite dimensional function space $V$, and approximating it on a finite dimensional sub-space $V_{h}$. In our case the subspace will be piecewise polynomial functions defined on our subdivision.

\item Projecting the boundary conditions onto the finite dimensional space.
\end{enumerate}

Once the mathematical context of the Finite Element has been outlined we will move on to finding the weak formulation of the Heat Equation. With this done we will consider the finite dimensional subspace and the elements that will be used in our discretisation. We will close the chapter will a discussion of other possible choices for elements.

\subsection{Sobolev Spaces} \label{subsec:Sobolev}

We begin by stating the notation to be used in this dissertation and also stating some of the relevant definitions and results from functional analysis.

\theoremstyle{definition}
\begin{definition}{Continuous Functions}
Let $\Omega \subset \mathbb{R}^n, n \geq 1$ be an open bounded set. 
\begin{itemize}
\item We define $C(\Omega)$  the set of all real-valued and continuous functions defined on $\Omega$.
\item For $m \geq1$ we write $C^m(\Omega)$ to denote the set of m times continuously differentiable functions  i.e. that $C^m(\Omega) = \{f \in C(\Omega) : f^{(k)} \in C(\Omega)\quad \forall k \leq m\} $.
\item Let $C^{\infty}_0(\Omega) $ denote the set of all infinitely many times differentiable that vanish on the boundary of $\Omega$.
\end{itemize}
\end{definition}


\theoremstyle{definition}
\begin{definition}{Function Support}
Let $\Omega \subset \mathbb{R}^n, n \geq 1$ be an open bounded set. 

\end{definition}

\theoremstyle{definition}
\begin{definition}{$L^p$ Space}

$\|u\|_{L_2(\Omega)} := \mathlarger{\Big(\int_\Omega  { |u|^p } dx\Big)}^{\frac{1}{p}} < \infty$

\end{definition}


\theoremstyle{definition}
\begin{definition}{Multi Index}

$\alpha = (\alpha_1, \alpha_2, ... \alpha_n) \in \mathbb{N}^n \quad \quad   |\alpha| = \alpha_1 + ... \alpha_n$

Define the Multi Index 

$D^{\alpha} = \Big(\frac{\partial}{\partial x_1}\Big)^{\alpha_1}...\Big(\frac{\partial}{\partial x_n}\Big)^{\alpha_n} = $
$\frac{\partial^{|\alpha|}}{\partial x^{\alpha}_1 ... \partial x^{\alpha}_n}$

We can now generalise our earlier definition of continuous differentiable functions to an analogous idea with the multi index which will be relevant to partial differentiation.

$C^m(\Omega) = \{f \in C(\Omega) : D^\alpha f \in C(\Omega)\quad \forall |a| \leq\leq m\} $

\end{definition}


\theoremstyle{definition}
\begin{definition}{Weak Derivative}

Say $\psi_\alpha$ is a locally integrable 

$\mathlarger{\int_\Omega  \psi_\alpha(x) \cdot v(x) dx} = (-1)^{\alpha} \mathlarger{\int_\Omega  u(x) \cdot D^{\alpha} v(x) dx} \forall v \in C^\infty_0(\Omega)$

\end{definition}

\theoremstyle{definition}
\begin{definition}{Sobolev Space}

$H^{m}(\Omega) = \{ u \in L_2(\Omega) : D^{\alpha} u \in L_2(\Omega), |\alpha| \leq m \}$

$$\|u\|_{H^m(\Omega)} :=  \Big(\mathlarger{\sum_{|\alpha|\leq m}}\|D^{\alpha}u\|^2_{L_2(\Omega)} \Big)^{\frac{1}{2}} $$


\end{definition}

The Lax Milgram theorem shows shows the existence and uniqueness of a solution to the PDE. 


\begin{theorem}[Lax Milgram]
\label{Lax Milgram}
Let V be a real Hilbert Space with associated norm $\|\dot{•}\|$. Also let $a(\dot{•}, \dot{•})$ be a bilinear functional on V x V with:

(a) $a(\cdot, \cdot)$ is coercive, i.e. there exists a positive constant $c_0 \forall v \in V |a(v, v)|  \geq c_0 \|v\|^2_V$ 

(b) $a(\cdot, \cdot)$ is continuous i.e. there exists a positive constant $c_1 \forall v, w \in V |a(v, w) \leq c_1 \|v\|_V \|w\|_V$ 

and for $\ell(\cdot)$ a bilinear functional also on V such that 

(c) $\ell(\cdot)$ is continuous i.e. there exists a positive constant $c_2 \forall v \in V |\ell(\cdot)|  \leq c_2 \|v\|_V$

Then there $\exists u \in V: a(u, v) = \ell(v) \forall v \exists V$ and u is unique.
\end{theorem}

This version of the theorem is Weaker than the refinement proved by Babuska but it sufficient for our purposes. The refined theorem is available f in most textbooks on the FEM.




\clearpage


\subsection{The Weak Formulation} \label{subsec:Weak Formulation}

Now we have defined the mathematical context let us solve a problem via the FEM. Take the stationary heat equation.

Consider the steady state one dimensional Heat Equation with conductivity co-efficient of one. 



\begin{subequations} 
\label{eq:Steady Diffusion1} 
\begin{align}
-u'' &= f		\quad \quad  x \in [0, 1] \\  	
u(0) &= 0 \\
u(1) &= 1
\end{align}
\end{subequations}

The first step of solving this and any PDE by the FEM is to find a suitable weak formulation. This is given by multiplying both sides of the equation by a test function in the Sobolev space $v \in H^1_0(0, L)$

\begin{equation*}
\int_0^L  u'' v  dx = \int_0^L  f v dx 	\quad \quad  dx \forall v \in H^1(0, L)
\end{equation*}

Integrating by parts on the left leads us to our weak formulation of \ref{eq:Steady Diffusion1}

\begin{problem}{Weak Formulation} \label{prob:Weak Formulation Elliptic}
Find $\quad u \in H^1(0, 1)$:
\begin{align*}
\int_0^L  u' v'  dx =   \int_0^L  f v dx  \quad \quad  dx \forall v \in H^1(0, L)
\end{align*}
$\forall v \in H^1(0, L)$
\end{problem}

Define 
\begin{equation*}
\int_0^L  u' v'  dx = a(u, v)  	
\int_0^L  f v dx  =  \ell(v)
\end{equation*}

This gives us the linear and bilinear form with can be used to show by \ref{Lax Milgram} that $u$ exists and is unique. However, our test function has been chosen simply some function in the Sobolev space which has infinitely many functions defined within it and is described as infinite dimensional. 

To find the FEM approximation to \ref{Steady Diffusion1} we look for a solution in a finite dimensional subspace of $V\subset H^1(0, L) $. The subspace we chose consists of continuous piecewise linear functions of defined on some given subdivision of $\Omega$. Then we can restate the weak formulation as the following approximation:

\begin{problem} \label{prob:Approx1}
Find $\quad u_h \in  V\subset H^1(0, 1) \quad : \quad a(u_h, v_h) =  \ell(v_h) \quad \forall v_h \subset V.$
\end{problem}



\subsection{Spatial Discretization} \label{subsec:Subspace}



Take a simple Heat Equation:



\begin{subequations} 
\begin{align}
  \dot{u} - u'' = & f \quad x \in [0, L], \quad t \in (0, T]  \\ \label{eq:Simple Heat}
  u(0, t) & = u(L, t) = 0\\
  u(x, 0) & = u_0(x)   
\end{align}
\end{subequations}

Multplying $\dot{u} - u'' = f$ by a test function  $v \in H^1_0(0, L)$  on both sides gives:

\begin{equation}
  \int_0^L \dot{u} v dx  - \int_0^L  u'' v  dx =   \int_0^L  f v  dx   
\end{equation}

Integrating by parts yields:

\begin{equation}
  \int_0^L \dot{u} v dx  + \int_0^L  u'' v  +u(0)v(0) - u(L)v(L) dx =   \int_0^L  f v  dx   
\end{equation}

\begin{equation}
  \int_0^L \dot{u} v dx +  \int_0^L  u' v'  dx =   \int_0^L  f v  dx    
\end{equation}

The weak formulation of our problem is now given by:

\begin{problem}{Weak Formulation}


Find $\quad u(\cdot, t) \in H^1_0(0, L)$:
\begin{align*}
 \int_0^L \dot{u} v dx +  \int_0^L  u' v'  dx =   \int_0^L  f v  dx& \\
  \int_0^L  u(x, 0) v dx = \int_0^L u_0(x) v dx&  
\end{align*}
$\forall v \in H^1_0(0, L)$

\end{problem}

\newpage


\section{Fully Discrete Problem} \label{sec:Fully Discrete}

\subsubsection{Implicit Euler} \label{subsubsec:Implicit Euler}


\subsubsection{Impementation} \label{subsubsec:Implementation}

\newpage

\section{A Posteriori Error Analysis} \label{sec:Errors}

When a continuous problem such as the Heat Equation is discretised Finite Element Methods and solved numerically an error is incurred. Some knowledge about the error is available in advance of the implementation of a specific FEM approximation. This type of error analysis, so called a priori, is valuable for the analysis of convergence and other proofs that may be important about an FEM scheme in general. Although useful in these applications it is rare that an a priori error estimate accurately quantifies the error present in a numerical solution. As such in a practical context the a priori error cannot be used to indicate the quality of a solution. 

In contrast, a posteriori error analysis uses the computed numerical solution itself to analyses the error. Where a posteriori error estimates are available and accurate this has the obvious advantage that it describes the error in the approximation without the need for an analytical solution. This replaces the heuristics and guess work that had previously guided the error analysis of the FEM approximation in settings where no analytical solution was available. 

Since the initial pioneering methods of Babuska \cite{babuska1978posteriori} \cite{babuska1981posteriori} in this field many a posteriori error methods have been presented. Many of these have been shown to be effective in context. The emphasis in the literature has moved now from the discovery of new a posteriori estimates to the testing of the limitations of existing a posteriori estimates. All this is done primarily with the goal of validating the extensively used self-adapting FEM packages.

\subsubsection{Gradient Recovery} \label{subsubsec:KK}

The FEM provides an approximation to an unknown function, however the gradient of the function may also be of particular interest also. The gradient in the approximation is discontinuous over element boundaries but can be smoothed out by post-processing. In fact under certain circumstances the smoothed gradient is closer to the gradient of the true solution \cite{ainsworth65001posteriori}. This leads fairly directly one of the simpler a posteriori error estimates which is measuring the gradient before and after this post-processing.

Consider again the steady state one dimensional Heat Equation with conductivity co-efficient of one \ref{eq:Steady Diffusion1} and here restating its weak formulation for ease.

\begin{problem}{Weak Formulation}

Find $\quad u(\cdot, t) \in H^1_0(0, 1)$:
\begin{align*}
\int_0^L  u' v'  dx =   \int_0^L  f v dx  \quad \quad  dx \forall v \in H^1_0(0, L)
\end{align*}
$\forall v \in H^1_0(0, L)$
\end{problem}

With
\begin{equation*}
\int_0^L  u' v'  dx = a(u, v)  	
\end{equation*}

It is clear that the bilinear form is symmetric i.e.  a(w, v) = a(v, w) and it can be shown that it is coercive as described in the Lax Milgram theorem \ref{Lax Milgram}. Hence, $a(w, v)$ is an inner product on $H^1_0(0, L) (w, v)_a$.

We can then define the associated energy norm.

\begin{equation}
\|w\|_a = |(w, w)_a|^\frac{1}{2}
\end{equation}

If we measure the true error in the energy norm we get.

\begin{equation}
\|u-u_h\|_a^2 = \int_0^L  u' - u_h'  dx
\end{equation}

Which is the $L_2$ norm of of the difference between the gradient .

If we had the true gradient we would now be able to measure the true error in the energy norm. As that is unavailable we will use an estimate of the true gradient which will be obtained by some suitable post- processing of the FEM approximation to the solution. Let the post processed FEM approximated gradient be denoted by $G_h(u_h)$ then the error indicator for \ref{eq:Steady Diffusion} is given by:

\begin{equation}
\eta^2 = \int_0^L  |G_h(u_h) - u_h|^2  dx
\end{equation}

Clearly there are as many different error indicators as there are ways of processing the FEM approximation to the gradient.

Consider an FEM approximation to an PDE of form \ref{eq:Simple Heat} at an arbitrary time step.

\begin{center}
\begin{tikzpicture}

\draw[thick,->] (0,0) -- (10,0);
\draw[thick,->] (0,0) -- (0,2.5);

\draw[blue,thick,dashed] (0,0) -- (2,2.2) -- (4,2.8) -- (8,0);

\end{tikzpicture}
\end{center}

The gradient of this approximation is a discontinuous step-like function, undefined at the nodes of the space mesh.

\begin{center}
\begin{tikzpicture}

\draw[thick,->] (0,2) -- (8,2) ;
\draw[thick,->] (0,0) -- (0,6) ;

\draw[blue,thick] (0,3.1) -- (2,3.1);
\draw[blue,thick] (2,2.3) -- (4,2.3);
\draw[blue,thick] (4,1.3) -- (8,1.3);

\filldraw (2,2.7) circle (2pt);
\filldraw (4,1.8) circle (2pt);


\end{tikzpicture}
\end{center}

Intuitively a better approximation would be given by a function defined in the whole domain. A simple but logical one to use could be a continuous piecewise linear function as we have the architecture in our code to linearly interpolate sets of points. We take the midpoint of the FEM approximated gradient where currently our gradient is undefined and hence where we expect post-processing to have the greatest effect. 

More formally results exist to support this choice. In the case of a uniform partition the "super-convergence" of the centroid of two elements is shown by \cite{zlamal1978superconvergence}. Though true "super-convergence" may not be assured on our more general partition our method none-the-less should be an improvement on the untreated gradient. 

A we also need to decide what to do with the elements at each side as our current method will not produce points for on the boundaries. We will choose to have the interpolant take the value of the gradient in the centre of the elements.



\begin{center}
\begin{tikzpicture}

\draw[thick,->] (0,2) -- (8,2) ;
\draw[thick,->] (0,0) -- (0,6) ;

\draw[blue,thick] (0,3.1) -- (2,3.1);
\draw[blue,thick] (2,2.3) -- (4,2.3);
\draw[blue,thick] (4,1.3) -- (8,1.3);

\filldraw (1,3.1) circle (2pt);
\filldraw (2,2.7) circle (2pt);
\filldraw (4,1.8) circle (2pt);
\filldraw (6,1.3) circle (2pt);

\draw[blue,thick,dashed] (0,3.5) --(1,3.1) --(2,2.7) -- (4,1.8)  -- (6,1.3) -- (8,0.8);


\end{tikzpicture}
\end{center}

\subsubsection{Implementation} \label{subsubsec:KK Implementation}

To make the most use of our existing member functions in the SpaceMesh class we should make sure our gradient recovery vector match the space nodes. This will allow the interpolant to be completely defined by the new vector and a reference to a grid. This means that the first and last point though should be defined on the boundary. In practise this means hard coding an interpolant over the first and last element and taking its value at the boundary.

\begin{lstlisting}[language=C++]
void GeneralHeat::GradientRecoveryFunction( SpaceMesh& relevantMesh,
       std::vector<double>& gradvec, std::vector<double>& gradrecovery ) {
    	
    	//(x_0, y_0) (x_1, y_1) define the line segment to be hard coded
    double x_0 = 0.5*(relevantMesh.ReadSpaceNode(1)+
    				relevantMesh.ReadSpaceNode(0));
    double y_0 = gradvec.at(0);
    double x_1 = relevantMesh.ReadSpaceNode(1);
    double y_1 = 0.5*(gradvec.at(1)+gradvec.at(0));

    gradrecovery.push_back(y_0+(relevantMesh.ReadSpaceNode(0)-
    						x_0)*(y_1-y_0)/(x_1-x_0));
		
		//take the midpoint of the two gradients where undefined
    for(int i = 0; i<relevantMesh.meshsize()-1; i++)
    {
        gradrecovery.push_back(0.5*(gradvec.at(i)+gradvec.at(i+1)));
    }
		
		//the hardcoding needs to happen on both sides
    x_0 = relevantMesh.ReadSpaceNode(mpsmesh.meshsize()-1);
    y_0 = gradrecovery.back();
    x_1 = 0.5*(relevantMesh.ReadSpaceNode(relevantMesh.meshsize())			   
    +relevantMesh.ReadSpaceNode(relevantMesh.meshsize()-1));
    y_1 = gradvec.back();

    gradrecovery.push_back(y_0+
    (mpsmesh.ReadSpaceNode(mpsmesh.meshsize())-x_0)*(y_1-y_0)/(x_1-x_0));
}

\end{lstlisting}



\newpage

\section{Design of Adaptive Algorithms} \label{sec:Adaptive}

\newpage

\section{Some Numerical Examples} \label{sec:Examples}

\newpage

\section{Object oriented design patterns} \label{sec:my2}

Graphics can be included. Figure \ref{fig:bsd} shows an example.
Learn about floats and pictures in the \LaTeX\ wikibook to place
the figures at the right place in the end.
%
\begin{figure}[h]
 \begin{center}
 \end{center}
 \caption{Oh look, something happens here !}
 \label{fig:bsd}
\end{figure}

\newpage

\section{Conclusions} \label{sec:conclusions}

Further help on \LaTeX\ can be found easily on the internet. The \LaTeX\
wikibook\footnote{\tt http://en.wikibooks.org/wiki/LaTeX} contains a lot.
For instance you would find there how to type theorems and proofs nicely.
Or how to include source code written in some programming language like
matlab. There are long lists available with all sorts of common
mathematical symbols like $\xi$, $\nabla$, $\infty$, $\log$, $\iff$, etc.

\newpage

\appendix

\section{Raw data} \label{app:rawdata}

This could contain a table of interesting numbers
\begin{center}
  \begin{tabular}{r|cccccc}
    $n$   & 1 & 2 & 3 & 4 & 5 & 6 \\ \hline
    $F_n$ & 1 & 1 & 2 & 3 & 5 & 8 \\
    $B_n$ & $\tfrac{1}{2}$ & $\tfrac{1}{6}$ & 0 & $-\tfrac{1}{30}$ & 0 &  $\tfrac{1}{42}$ \\
    $p_n$ & 2 & 3& 5& 7 & 11 & 13 \\
  \end{tabular}
\end{center}

Material that needs to be included but would distract from the main
line of presentation can be put in appendices.
Examples of such material are raw
data, computing codes and details of calculations.


\section{Calculations for section \ref{sec:my1}} \label{app:calculations}

In this appendix we verify equation \eqref{eq:myeq1}.

\newpage

	\bibliography{References}
	\bibliographystyle{plain}

\end{document}
